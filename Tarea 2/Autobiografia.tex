\documentclass[12pt, letterpaper]{article}
\usepackage[utf8]{inputenc}
\usepackage[spanish]{babel}
\usepackage[usename]{xcolor}

\title{Minimalismo y un poco más de mi}
\author{Valencia Sánchez Francisco David}
\date{12 de Septiembre 2022}

\begin{document}

\maketitle

\section{\huge{Academia} }
\subsection{\Large{Pasado}}
Vengo del colegio de ciencias y humanidades plantel Naucalpan, aproximadamente me hacía 1 hora con 45 minutos, es el CCH que me quedaba más cerca de mi casa, actualmente estoy viviendo en el Estado de México, específicamente en el Municipio de Tultitlan, como era del turno de la mañana, me tenía que levantar a las 4 de la mañana para salir de mi casa 5:30, en tiempo de pandemia usaba un transporte especial que nos llevaba directamente al CCH.
\subsection{\Large{Actualidad}}
La Facultad de Ciencias no queda cerca de mi domicilio, como vengo del estado a la ciudad aproximadamente me hago 2 horas con mucho tráfico, enfrente de mi casa pasa la combi que me deja en el metro Indios verdes, llegando al metro Indios Verdes me voy toda la línea hasta Universidad, saliendo me puedo ir caminando 10 minutos hasta la facultad.
\section{\huge{Hobbie}}
\subsection{\Large{Orquesta} }
Desde niño siempre me interesó el aprender a tocar un instrumento musical, llegada la edad de 12 años me decidí por tocar la batería, como soy del Estado de México, mi madre sabía que los mejores músicos salen del Municipio de Tultepec, un diciembre me llevo a la Casa de Cultura y entre al instrumento de las percusiones, meses después ingresé en la Orquesta Sinfónica, actualmente ya casi me he retirado, pero aun practico mi instrumento 1 o 2 veces por semana.
\subsection{\Large{Cubos Rubik}}
Mi pasión y gusto por los cubos de rubik nació en YouTube, explorando uno llega a rincones un poco extraños, afortunadamente para mi solo llegue a videos de cubos y personas resolviéndolos a gran velocidad y distintas clases, no solo el clásico 3x3, me compre un cubo de rubik hace unos 2 años después de ponerme el reto de armarlo, actualmente puedo resolver distintas clases de cubos, y el único que puedo resolver en el menor tiempo posible es el clásico, el tiempo que dedico a mi Hobbie es de 20 minutos al día, puede variar, depende de la carga de trabajo que tenga en el momento.
\section{\huge{Musica}}
\subsection{\Large{Musica Clasica}}
Independiente del gusto que adquirí en la orquesta por la música clásica, es el género favorito de mi papá, desde niño crecí escuchando este género y con el tiempo le fui tomando gusto y cariño, cuando llegue a estudiar música fueron las bases, la instrumentación me ayuda mucho a concentrarme en cosas, no me desvío o divago en la letra, además que me transmite mas que una letra en cuestión sentimental.
\subsection{\Large{Boleros}}
Muy alejado de la música clásica mi madre tenía gustos muy variados, desde niño siempre me gustó el sonido de la guitarra en el género del bolero, es eso que le da la magia, digamos que es la salsa de los tacos, me gusta el sabor romántico y triste del bolero, sin olvidar que para mi es un género muy íntimo algo que debes apreciar en un ambiente adecuado y con las personas correctas.
\section{\huge{Minimalismo}}
\subsection{\Large{¿Qué es el minimalismo?}}
\textcolor{green}{El minimalismo es una corriente artística que sólo utiliza elementos mínimos y básicos. Por extensión, en el lenguaje cotidiano, se asocia el minimalismo a todo aquello que ha sido reducido a lo esencial y que no presenta ningún elemento sobrante o accesorio.}
\subsection{\Large{Por qué soy minimalista}}
El minimalismo me ayuda demasiado ya que soy una persona que le cuesta tener las cosas en orden, ademas tengo una filosofia de vida en la cual dejo todos mis apegos materiales para tener presente las cosas que realmente me importan, la vida es mas sencilla si nos limitamos a lo realmente importante, ademas la estetica me agrada a la vista 

\end{document}
