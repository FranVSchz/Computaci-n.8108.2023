\documentclass[12pt, letterpaper]{article}
\usepackage[utf8]{inputenc}
\usepackage{amsmath} 
\usepackage{amssymb} 
\usepackage{mathrsfs}
\usepackage{latexsym}
\usepackage{dsfont}
\usepackage{xcolor}
\usepackage{fancyhdr}
\usepackage{enumitem}
\usepackage{amssymb}
\usepackage{emoji}
\pagestyle{fancy}
\fancyhf{}
\lhead{\leftmark}
\lfoot{Valencia Sánchez Francisco David}
\rfoot{\thepage}



\begin{document}
{\centering{{\large{\textls{\textbf{{{Ecuaciones para la vida}}}}}}\\
{{\large{{Valencia Sánchez Francisco David}}}}\\ 
\normalsize{27 Octubre 2022}}\\}



%-------------------------------------------------------------------
\section{Fisica}

\begin{itemize}
    \item [\star]Segunda ley de Newton, La Segunda Ley de Newton también conocida como Ley Fundamental de la Dinámica, es la que determina una relación proporcional entre fuerza y variación de la cantidad de movimiento o momento lineal de un cuerpo. Dicho de otra forma, la fuerza es directamente proporcional a la masa y a la aceleración de un cuerpo.
\end{itemize}

   $$\textcolor{red}{ \Vec{F} = m\Vec{a} }$$

%-------------------------------------------------------------------
\begin{itemize}
\item[\bigstar]Ecuacion cinematica para un cuepo en caida libre, Fórmulas de caída libre Existen diversas fórmulas para el tema de caída libre, sin embargo es importante diferenciar unas de otras ya que despejando algunas variables se nos generará otra fórmula y así sucesivamente. Considerando a la gravedad 
\end{itemize}
    $$ y - y_0 = V_0 - \frac{1}{2} g t^2 $$
    $$ v_f = v_0 - g t $$
    $$ y - y_0 = \frac{1}{2} (v + v_0) t $$
 %--------------------------------------------
\begin{itemize}
\item[\textsection] Ley de gravitación universal, La ley de gravitación universal es una ley física clásica que describe la fuerza o interacción gravitatoria entre distintos cuerpos con masa
\end{itemize}
    $$ \Vec{F} = G \frac{m_1 * m_2}{d^2}$$
%---------------------------------------------  
\begin{itemize}
\item [ \clubsuit] Equivalencia entre masa y energía, o ecuación de equivalencia masa-energía, es la relación entre masa y energía en el marco de reposo de un sistema, donde los dos valores difieren solo por una constante y las unidades de medida. El principio está descrito por la famosa fórmula del físico Albert Einstein
\end{itemize}
    $$ E = m c^2 $$
    
%----------------------------------------------
\begin{itemize}
\item[\spadesuit] La Ecuacion de Shrödinger dependiente del tiempo, 
La forma de la ecuación de Schrödinger depende de la situación física. La forma más general es la ecuación dependiente del tiempo, la cual describe un sistema que evoluciona con el paso del tiempo
\end{itemize}
$$ i \hbar \frac{\partial}{\partial t} \Psi (r,t) = \vec{H} \Psi (r, t)$$
%----------------------------------------------
\begin{itemize}
\item [\star]Ecuación Independiente del tiempo(partícula simple no relativista), Una conocida aplicación, es la ecuación de Schrödinger no relativista para una partícula simple moviéndose en un campo eléctrico (pero no en uno magnético):
\end{itemize}
$$ E \Psi (r) = [\frac{-\hbar^2}{2\mu}\nabla^2 + V(r)] \Psi (r)$$

%----------------------------------------------

\begin{itemize}
\item [\bigstar]Tercera ley de Kepler, Donde, T es el período orbital (tiempo que tarda en dar una vuelta alrededor del Sol), a  la distancia media del planeta con el Sol y la constante de proporcionalidad.
Estas leyes se aplican a otros cuerpos astronómicos que se encuentran en mutua influencia gravitatoria, como el sistema formado por la Tierra y el sol.
\end{itemize}

$$T = 2\pi \sqrt{\frac{r^3}{GM_E}}$$

%----------------------------------------------
\begin{itemize}
\item[\textsection] Ecuación de continuidad, mecánica de fluidos
Para fluido incompresible con densidad constante se requiere que el elemento de fluido tenga densidad constante al moverse solo por una línea de corriente, o sea que la derivada sustancial con respecto al tiempo sea cero.
Forma diferencial
\end{itemize}

$$ \frac{\partial\rho}{\partial t} + \nabla * (\rho v) = 0$$

%----------------------------------------------
\begin{itemize}
\item[\clubsuit] Fórmula del periodo de un péndulo 
El periodo de un péndulo simple depende de su longitud y de la aceleración debido a la gravedad. El periodo es completamente independiente de otros factores, como masa y desplazamiento máximo
\end{itemize}

$$ T = 2\pi \sqrt{\frac{L}{g}} $$
%----------------------------------------------

\begin{itemize}
    \item[\spadesuit] La transformada de Fourier, divide el tiempo en varias frecuencias y ondas simples como un prisma desglosa la luz en varios colores.
\end{itemize}

$$ f(x) = \int_{-\infty}^{\infty} F(u) e^{2\pi ux} dx$$

%----------------------------------------------
\begin{itemize}
    \item [\star] La norma de un vector, es su distancia nos es de utilidad en mecanica, este es en plano cartesiano de dos dimensiones 
\end{itemize}
    $$ |\vec{A}| = \sqrt{a_{x}^{2} + a_{y}^{2}} $$
%----------------------------------------------
\begin{itemize}
    \item [\bigstar] Velocidad media, la velocidad media de un objeto se define como la distancia recorrida por un objeto dividido por el tiempo transcurrido.
\end{itemize}

    $$ \vec{v} = \frac{\Delta x}{\Delta t}$$
%----------------------------------------------------------
\begin{itemize}
    \item[\textsection]La Ley de Hooke es una ley de la física que determina la deformación que sufre un cuerpo elástico mediante una fuerza. La teoría establece que el estiramiento de un objeto elástico es directamente proporcional a la fuerza que se le aplica. 
\end{itemize}
    $$\vec{F}=-k\vec{\Delta x}$$
%----------------------------------------------------------
\begin{itemize}
    \item[\clubsuit] Crntro de masa, es el punto donde una partícula de masa igual a la masa total del sistema sometida a la resultante de las fuerzas que actúan sobre el mismo.
\end{itemize}
    $$ x_c = \frac{m_1 * x_1 + m_2 * x_2 }{ m_1 + m_2}$$
%--------------------------------------------------------------
\section{Geometria Analitica}
\begin{itemize}
    \item [\star] Teorema de Pitagoras, área del cuadrado cuyo lado es la hipotenusa (el lado opuesto al ángulo recto) es igual a la suma de las áreas de los cuadrados de los otros dos lados.
\end{itemize}

    $$  \textcolor{red}{a^2 + b^2 = c^2} $$
    
%----------------------------------------------------------------    
\begin{itemize}
    \item[\bigstar] Distancia entre dos puntos, la distancia entre dos puntos equivale a la longitud del segmento de recta que los une, expresado numéricamente. Distancia entre dos puntos. Dados dos puntos cualesquiera
\end{itemize}
    $$ d(A,B) = \sqrt{(x_2 + x_1)^2 - (y_2 + y_1)^2 }$$
%-----------------------------------------------
\begin{itemize}
    \item [ \textdaggerdbl ] Ecuación general de la recta, La ecuación general de la recta describe el comportamiento de todas las rectas existentes en el plano cartesiano.
No importa la recta que se trace siempre va a cumplir con esta ecuación
\end{itemize}

    $$ Ax + By + C = 0 $$
    
%-------------------------------------------------------------------

\begin{itemize}
    \item [\bigstar] Formula de la pendiente, La pendiente se calcula como la relación entre el cambio en el eje y y el cambio en el eje x. La pendiente de una línea recta describirá el ángulo de inclinación de la línea con respecto a la horizontal, tanto si sube como si baja. Cuando la línea no sube ni baja, su pendiente será cero.
\end{itemize}

    $$ m = \frac{y_2 - y_1}{x_2 - x_1}$$
    
%-------------------------------------------------------------------
    
\section{Calculo}

\begin{itemize}
    \item [\textparagraph] Defincion de Derivada,  La derivada es el resultado de un límite y representa la pendiente de la recta tangente a la gráfica de la función en un punto
\end{itemize}

    $$\textcolor{red}{\lim_{h \rightarrow 0} {\frac{f(a+h)-f(a)}{h}}}$$
    
%-------------------------------------------------------------------
\begin{itemize}
    \item [ \clubsuit] Teorema del binomio

\end{itemize}

    $$ (a + b)^n = \sum_{j=0}^{n} = \binom{n}{j} a^{n - j} n^j $$
%--------------------------------------------------------------------


    
%-----------------------------------------------
\section{Algebra}
\begin{itemize}
    \item[\star] Un binomio al cuadrado es un polinomio de dos términos que se encuentra elevado a la potencia de dos y se expresa de esta forma, una positiva y una negativa
 
    $$ (a + b)^2 =  a^2 + 2ab +b^2 $$
    
    $$ (a - b)^2 =  a^2 - 2ab +b^2 $$
\end{itemize}
%---------------------------------------------------------
\begin{itemize}
    \item[\bigstar]  Fórmula general para resolver las ecuaciones de segundo grado.
\end{itemize}

    $$\textcolor{red}{ x = \frac{b \pm \sqrt{b^2 - 4ac}}{2a}}$$

%----------------------------------------------


%----------------------------------------------




\end{document}
